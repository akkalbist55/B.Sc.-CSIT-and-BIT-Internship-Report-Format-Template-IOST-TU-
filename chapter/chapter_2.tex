\chapter{Organization Details and Literature Review}
\justifying

\section{Introduction to Organization}
The internship was carried out at \textbf{[Organization Name]}. 
The organization has a rich history and operates with a clear mission, vision, and core values. 
It plays a significant role in the industry and its activities are directly relevant to the internship project. 
The organization provides an environment conducive to learning practical skills and applying academic knowledge.

\section{Organizational Hierarchy}
The organization has a structured hierarchy consisting of multiple departments and management levels. 
A hierarchical chart illustrates the reporting relationships and departmental structure. 
The internship was conducted in the \textbf{[Department Name]} department, which plays a key role in the organization's operations.

% Example figure (optional)
% \begin{figure}[H]
% \centering
% \includegraphics[width=0.8\textwidth]{images/org_chart.png}
% \caption{Organizational hierarchy of the internship organization.}
% \label{fig:org_hierarchy}
% \end{figure}

\section{Working Domains of Organization}
The organization operates in several domains, including software development, network services, and database management. 
Activities relevant to the internship project were primarily in the areas of \textbf{[Specify relevant domains]}.

\section{Intern Department Details}
The internship was performed in the \textbf{[Department/Unit Name]}. 
This department is responsible for \textbf{[Department Functions]}. 
The team includes \textbf{[Number/Names of team members]} and employs technologies such as \textbf{[Technologies/Processes]}. 
The workflow and processes relevant to the project are described in the methodology chapter.

\section{Literature Review}
Undergraduate research and internship experiences play a crucial role in enhancing students’ academic development, practical skills, and professional identity. Studies have shown that such experiences positively influence students’ understanding of research processes, problem-solving abilities, and career motivation. In particular, the evaluation of undergraduate research internships highlights strong alignment between students’ learning outcomes and faculty expectations, emphasizing mentorship as a key factor in successful research engagement \parencite{kardash2000evaluation}. 

Modern internships increasingly involve advanced technologies, and students are expected to develop competencies that bridge theoretical knowledge with real-world applications, especially in interdisciplinary and technology-driven domains \parencite{goth2025foundational}.

Recent research trends demonstrate the growing importance of machine learning, data analytics, and artificial intelligence across diverse application areas such as cybersecurity and smart cities. Scientometric analyses reveal a rapid global expansion of machine learning–based cybersecurity research, underscoring its relevance for academic training and industry readiness \parencite{razzaq2025advancing}. Similarly, systematic reviews in smart city development highlight the central role of data mining and machine learning in enabling sustainable and intelligent urban systems \parencite{souza2019data}. These evolving technological demands expose significant competency gaps between higher education curricula and industry expectations, particularly in AI-related skills. Addressing this mismatch requires curriculum alignment, practical training, and industry-oriented internships, as emphasized by multinational studies on engineering and IT education reform \parencite{alhazmi2026aligning}.
